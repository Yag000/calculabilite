
\section{Machines de Turing}

\subsection{Définition}

\begin{definition}[Machine de Turing]
	Étant donné un alphabet $\Sigma$, une \textbf{machine de Turing} est un 6-uplet $M = (Q, \Gamma, \delta, q_0, q_a, q_r)$ où :
	\begin{itemize}
		\item $Q$ est un ensemble fini d'états
		\item $\Gamma$ est un alphabet fini de symboles de ruban, et $\Sigma \subseteq \Gamma$
		\item $\delta$ est la fonction de transition
		      $$ \delta: \underbrace{Q}_{\text{État currant}} \times \underbrace{\Gamma}_{\text{Lit}} \to \underbrace{Q}_{\text{Nouveau état}} \times
			      \underbrace{\Gamma}_{\text{Écrit}} \times \underbrace{\{R, L, N\}}_{\text{Direction}} $$
		\item $q_0 \in Q$ est l'état initial
		\item $q_a \in Q$ est l'état d'acceptation
		\item $q_r \in Q$ est l'état de rejet
	\end{itemize}
\end{definition}

\begin{definition}[Configuration]
	Une \textbf{configuration} d'une machine de Turing est un triplet $(q, c, pos)$ où :
	\begin{itemize}
		\item $q \in Q$ est l'état courant
		\item $c$ est le contenu du ruban
		\item $pos$ est la position de la tête de lecture
	\end{itemize}
\end{definition}

Comment se déroule une exécution d'une machine de Turing pour un mot $w \in \Sigma^*$ ?

\begin{enumerate}
	\item On initialise le ruban
	      \begin{figure}[!htb]
		      \centering
		      % Inspired from https://tex.stackexchange.com/questions/49839/turing-machine-figure
		      \begin{tikzpicture}[every node/.style={block},
				      block/.style={minimum height=1.5em,outer sep=0pt,draw,rectangle,node distance=0pt}]
			      \node (A) {$w_0$};
			      \node (B) [left=of A] {$B$};
			      \node (C) [left=of B] {$\ldots$};
			      \node (D) [right=of A] {$\ldots$};
			      \node (E) [right=of D] {$w_n$};
			      \node (F) [right=of E] {$B$};
			      \node (G) [right=of F] {$\ldots$};
			      \node (T) [below = 0.75cm of A] {$q_i$};
			      \draw[-latex] (T) -- (A);
			      \draw (C.north west) -- ++(-1cm,0) (C.south west) -- ++ (-1cm,0)
			      (G.north east) -- ++(1cm,0) (G.south east) -- ++ (1cm,0);
		      \end{tikzpicture}
	      \end{figure}

	\item Si on est dans l'état $q$, que la lettre sous la tête de lecture est $a$ et que $\delta(q, a) = (q', b, d)$, alors on fait :
	      \begin{itemize}
		      \item On écrit $b$ à la place de $a$
		      \item On se déplace dans l'état $q'$
		      \item On déplace la tête de lecture dans la direction $d$
	      \end{itemize}
	\item Si on arrive dans l'état $q_a$ ou $q_r$, alors on arrête l'exécution. Si on arrive dans l'état $q_a$, alors on accepte le mot $w$ et si on arrive dans l'état $q_r$, alors on rejette le mot $w$.
\end{enumerate}


\begin{notation}
	Soit $M$ une machine de Turing, $w \in \Sigma^*$ un mot, alors on note $M(w)$ l'exécution de $M$ sur $w$. Cette exécution peut être :
	\begin{itemize}
		\item Acceptée : $M(w) = 1$
		\item Rejetée : $M(w) = 0$
		\item Bouclée : $M(w) = \bot$
	\end{itemize}
\end{notation}


\begin{remarque}
	Les automates s'injectent dans les machines de Turing, un automate est une machine de Turing qui se déplace tout le temps vers la droite.
\end{remarque}

\subsection{Machines de Turing non déterministes}


\begin{definition}[Machine de Turing non déterministe]
	De manière analogue aux automates la notion de \textbf{machine de Turing non déterministe} étend la définition d'une machine de Turing en
	permettent d'avoir plusieurs transitions pour un état donné. La différence se trouve donc dans le type de la fonction de transition :

	$$ \Delta: Q \times \Gamma \to \parts{Q \times \Gamma \times \{R, L, N\}}$$
\end{definition}

\begin{prop}
	Les machines de Turing déterministes et non-déterministes reconnaissent les memes langages.
\end{prop}

\begin{proof}
	L'idée derrière la preuve est d'explorer les branches de l'évaluation non déterministe de la machine. Cette exploration
	se fait en largeur, afin d'éviter de suivre une branche qui boucle alors qu'une autre branche acceptante aurait pu être explorée.

	La description de la machine a été étudiée en TD, et elle est également disponible dans la preuve du \cite[Theorem~3.16]{sipser}.
\end{proof}
