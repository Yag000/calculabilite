\section{Hiérarchie arithmétique}


%TODO: Add 
Formules arithmétiques du premier ordre sous forme prénexe.


\begin{definition}[Saut de Turing]
	$X$ est un langage,

	$X' = \setdef c {\phi_c^X(c) \text{ est défini}} = \setdef {\encode M} {M^X \text{ s'arrete sur } \encode M}$

	$\phi_{c \in \mots}$ : énumeration des fonctions calculables.

	$\phi_{c \in \mots}^X$ : énumeration des fonctions calculables relativement a $X$. Et donc $\phi_{\encode M}^X(w) = M^X(w)$
\end{definition}


\begin{exemple}
	$$\emptyset' = \setdef {\encode M} {M^{\emptyset} \text{ s'arrete sur } \encode M} \equivm \halt$$
\end{exemple}

\begin{exercice}
	Montrer que $\emptyset '$ est $\Sigma_1$-complet :
	\begin{enumerate}
		\item $\emptyset' \in \Sigma_1$
		\item $\forall L \in \Sigma_1, L \leqm \emptyset'$
	\end{enumerate}
\end{exercice}


\begin{lemma}
	Si $A$ est $C$-complet:  $\exists Y$ tel que $X$ est $re(Y)$ avec $Y \in C$ $\iff$ $X$ est $re(A)$.
\end{lemma}

\begin{theorem} [de post]
	Nous avons les résultats suivants :
	\begin{enumerate}
		\item
		      \begin{enumerate}
			      \item
			            \begin{eqnarray*}
				            L \in \Sigma_{n+1} &\iff& L \text{ est r.e. relativement  un langage }  \Pi_n  \\
				            &\iff& L \text{ est r.e. relativement  un langage }  \Sigma_{n}
			            \end{eqnarray*}

			      \item
			            \begin{eqnarray*}
				            L \in \Pi_{n+1} &\iff& L \text{ est co-r.e. relativement  un langage }  \Sigma_n  \\
				            &\iff& L \text{ est r.e. relativement  un langage }  \Pi_n
			            \end{eqnarray*}
		      \end{enumerate}

		\item Il existe un langage $\Sigma_n$ complet, noté $\emptyset^{(n)}$.
	\end{enumerate}
\end{theorem}

TODO

