\section{Théorèmes de récursion}


\begin{theorem}[d'itération / Smn / d'application partielle]\label{thm:it}
	Il existe une fonction calculable et totale $s$ \tlq
	$$\forall n,m,w, \phi_{s(m,n)}(w) = \phi_m(\encode {n,w})$$
	Si $m$ est le code d'un programme et $n$ un mot, alors $s(m,n)$ est le code de l'application partielle de $\phi_m$ à $n$.
\end{theorem}

\begin{theorem}[de point fixe]
	Si $ \fmots f$ \emph{calculable} et \emph {totale}. Alors il existe $e \in \mots$ \tq $\phi_e = \phi_{f(e)}$.
\end{theorem}


\begin{proof}
    Soit $G$ la machine : $(x,y) \to \letin e {\universal x x} {\universal e y}$.

	On pose $h(x) = s(\encode G, x)$ (le $s$ du théorème precedent). On a que $f \circ h$ est calculable et totale et notons son code $c$. Alors

	\begin{eqnarray*}
		\phi_{h(c)} (w) &=& \phi_{s(\encode G, c)} (w) \reason{par définition de $h$} \\
		&=& \phi_{\encode G} (\encode {c, w}) \reason{par \ref{thm:it}} \\
		&=& G (\encode {c, w}) \reason{correspondence énumeration machine }\\
		&=& \letin e {\universal c c} \universal e w \reason{par définition de $G$ }\\
		&=& \phi_{f \circ h (c)}(w) \reason{car $\universal c c =_{\ref{lem:univ}} \phi_c(c) =_{\ref{def:enum}} f \circ h (c)$}
	\end{eqnarray*}
	Donc $\phi_{h(c)} = \phi_{f \circ h (c)}$ et donc $h(c)$ est notre point fixe.
\end{proof}


\begin{theorem}[de récursion]
	Si $f : \mots \times \mots \to \mots$ est une fonction partielle et calculable. Alors il existe une machine $R$ qui calcule $\fmots r $ \tq
	$$ \forall w, r(w) =  f (\encode R, w)$$

	Autrement dit, $\exists e, \phi_e (w) = f (e, w)$
\end{theorem}


\begin{proof}
	Soit $M_f$ une machine qui calcule $f$ et $\fmots g, \ g(p) = s(\encode {M_f}, p)$
	Alors on applique le théorème de point fixe à $g$ et on a qu'il existe $e$ \tq $\phi_e(w)
		=\phi_{s(\encode{M_f},e)} (w) = \phi_{\encode {M_f}} (e,w) = f(e,w)$
\end{proof}


\begin{theorem}[de Rice]
	Toute propriété non triviale relative au langage reconnu par une machine de Turing est indécidable.

	Autrement dit, soit $L = \setdef {\encode M} {P(L_M)}$, avec $P$ une propriété non triviale, \ie $\exists M_1$ \tq
	$\encode {M_1} \in L$ et $\exists M_2$ \tq $\encode{M_2} \notin L$. Alors $L$ n'est pas décidable.
\end{theorem}


\begin{proof}
	Soit $L = \setdef {\encode M} {P(L_M)}$.
	Sans perte de généralité, supposons $P(\emptyset)$, \ie $\encode {M_{loop}} \in L$.
	On peut faire ceci car, dans le cas ou $\lnot P (\emptyset)$, alors on considère $\bar L$.

	Alors $\encode {M_{loop}} \in L \et \exists \encode{M_2} \notin L$, car $P$ est non triviale.

	Montrons que $\overline {\halt} \leqm L$.
	Soit $M$ une machine de Turing, pour tout $w$ on pose
	$$M' = \fun u {M(w); M_2(u)}$$
	Alors
	\begin{eqnarray*}
		M(w) = \bot &\iff& \lang M = \emptyset \\
		&\iff& M_L (M') = 1
	\end{eqnarray*}
	Donc $\overline {\halt} \leqm L$ et ainsi $L$ n'est pas décidable.
\end{proof}



\begin{prop}
	Il existe un Quine, \ie une machine $M$ \tq :
	$$\forall w, M(w) = \encode M$$
\end{prop}

\begin{proof}
	On applique le théorème de récursion avec la fonction $f(e, \_) = e$. Donc $\exists R, \, R(w) = f(\encode R, w) = \encode R$.
\end{proof}

