\documentclass{article}
\usepackage[utf8]{inputenc}
\usepackage[T1]{fontenc}

\usepackage[french]{babel}

\usepackage{amsmath}
\usepackage{amssymb} 
\usepackage{amsthm}  
\usepackage{dsfont}
\usepackage{mathrsfs}
\usepackage{mathtools}
\usepackage{mathpartir}


\usepackage{graphicx}
\usepackage{float}
\usepackage{geometry}
\usepackage{hyperref}

\usepackage{caption}

\usepackage{tikz}
\usetikzlibrary{shapes.geometric,calc,positioning}

\usepackage{xcolor}
\hypersetup{
	colorlinks,
	linkcolor={red!50!black},
	citecolor={blue!50!black},
	urlcolor={blue!80!black}
}


\usepackage{etoolbox}
\usepackage{stmaryrd}


\usepackage{fancyhdr}

\usepackage[pdf, singlefile]{graphviz}

\usepackage{algpseudocode,algorithmicx}


\usepackage{calculabilite}

\fancypagestyle{toc}{%
\fancyhf{}%
\fancyhead[L]{\rightmark}%
\fancyhead[R]{\thepage}%
}

\pagestyle{toc}


\begin{document}
\begin{titlepage}
	\newcommand{\HRule}{\rule{\linewidth}{0.5mm}}
	\center

	\HRule\\[0.4cm]

	\textsc{\Large calculabilité et complexité}\\[0.5cm]

	\HRule\\[1.5cm]

	{\large\textit{Auteur}}\\
	Yago \textsc{Iglesias}


	\vfill\vfill\vfill

	{\large\today}

	\vfill

\end{titlepage}

\tableofcontents

Ce document est un recueil de notes du cours de Calculabilité et Complexité de niveau M1, portant exclusivement sur la partie consacrée à la Calculabilité.
Il est basé sur les cours de M.~\textsc{Hugo Férée} à l'Université Paris Cité. Cependant, toute erreur ou inexactitude est de ma responsabilité.

Ce document a été rédigé principalement par \textsc{Yago Iglesias}, mais tout contributeur peut être retrouvé dans
la section contributeurs du répertoire \href{https://github.com/Yag000/calculabilite/graphs/contributors}{GitHub}.
Un remerciement particulier est adressé à \textsc{Erin Le Boulc’h} pour sa participation active à la rédaction et correction
de ce document.


\section{Introduction}


Dans ce cours, on s'intéresse à la notion de calculabilité, mais pour cela, il faut comprendre ce qu'est un problème.
D'une manière informelle, on peut voir un problème comme une fonction qui prend en entrée des données et retourne une réponse binaire.
On fait la distinction entre un problème et une fonction qui, au lieu de retourner une réponse binaire, renvoie un nouvel ensemble de données.
\begin{definition}
	Un \textbf{problème} est une fonction $f: \Sigma^* \to \{0, 1\}$, où $\Sigma$ est un alphabet fini.
\end{definition}

\begin{remarque}
	L'ensemble des problèmes est infini dénombrable et il est en bijection avec $\parts \Sigma^*$.
\end{remarque}

\begin{remarque}
	L'ensemble des fonctions (en utilisant la définition de fonction donnée) est l'ensemble des fonctions $f: \Sigma^* \to \Sigma^*$ qui est infini non dénombrable,
	et donc bien plus grand que l'ensemble des problèmes. De plus, l'ensemble des problèmes est un sous-ensemble de l'ensemble des fonctions à isomorphisme près.
\end{remarque}

Il nous manque maintenant la notion de programme pour pouvoir parler de calculabilité.

En 1900, David Hilbert a posé 23 problèmes mathématiques qu'il considérait comme les plus importants de son époque. L'un de ces problèmes, le 10e, consistait à trouver
une méthode (un nombre fini d'étapes) pour décider si une équation diophantienne a une solution entière. Cependant, aucune méthode n'a été trouvée, car il n'en existe pas.
Pouvoir faire ce genre de démonstration relève de la calculabilité. Une synthèse de cette preuve peut être retrouvée dans \cite{Hilbert10}.

Regardons quelques exemples de constructions qui sont normalement associées à la notion de programme :

\begin{itemize}
	\item Les langages de programmation comme Python, Java, C, etc. \circled 3
	\item Le $\lambda$-calcul \circled 3
	\item Les machines de Turing \circled 3
	\item Les automates $\iff$ les expressions régulières \circled 1
	\item Les automates à pile $\iff$ les grammaires hors-contexte \circled 2
	\item Les automates à 2 ou plusieurs piles \circled 3
	\item Les transducteurs
\end{itemize}

Ces constructions sont classées selon la hiérarchie de Chomsky \ref{fig:chomsky}.

\begin{figure}[!htb]
	\centering
	% From: https://tex.stackexchange.com/questions/484541/nested-ellipses-in-tikzpicture-chomsky-hierarchy
	\begin{tikzpicture}[font=\sffamily,breathe dist/.initial=2ex]
		\foreach \X [count=\Y,remember=\Y as \LastY] in
			{Régulières \circled 1, Hors contexte \circled 2, Context sensitive, Récursivement énumerables \circled 3}
			{\ifnum\Y=1
					\node[ellipse,draw,outer sep=0pt] (F-\Y) {\X};
				\else
					\node[anchor=south] (T-\Y) at (F-\LastY.north) {\X};
					\path let \p1=($([yshift=\pgfkeysvalueof{/tikz/breathe dist}]T-\Y.north)-(F-\LastY.south)$),
					\p2=($(F-1.east)-(F-1.west)$),\p3=($(F-1.north)-(F-1.south)$)
					in ($([yshift=\pgfkeysvalueof{/tikz/breathe dist}]T-\Y.north)!0.5!(F-\LastY.south)$)
					node[minimum height=\y1,minimum width={\y1*\x2/\y3},
							draw,ellipse,inner sep=0pt] (F-\Y){};
				\fi}
	\end{tikzpicture}
	\caption{Hiérarchie de Chomsky}
	\label{fig:chomsky}
\end{figure}


La thèse de Church-Turing est une hypothèse qui postule que la seule notion de problème décidables est \circled 3.



\section{Machines de Turing}

\subsection{Définition}

\begin{definition}[Machine de Turing]
	Étant donné un alphabet $\Sigma$, une \textbf{machine de Turing} est un 6-uplet $M = (Q, \Gamma, \delta, q_0, q_a, q_r)$ où :
	\begin{itemize}
		\item $Q$ est un ensemble fini d'états
		\item $\Gamma$ est un alphabet fini de symboles de ruban, et $\Sigma \subseteq \Gamma$
		\item $\delta$ est la fonction de transition
		      $$ \delta: \underbrace{Q}_{\text{État currant}} \times \underbrace{\Gamma}_{\text{Lit}} \to \underbrace{Q}_{\text{Nouveau état}} \times
			      \underbrace{\Gamma}_{\text{Écrit}} \times \underbrace{\{R, L, N\}}_{\text{Direction}} $$
		\item $q_0 \in Q$ est l'état initial
		\item $q_a \in Q$ est l'état d'acceptation
		\item $q_r \in Q$ est l'état de rejet
	\end{itemize}
\end{definition}

\begin{definition}[Configuration]
	Une \textbf{configuration} d'une machine de Turing est un triplet $(q, c, pos)$ où :
	\begin{itemize}
		\item $q \in Q$ est l'état courant
		\item $c$ est le contenu du ruban
		\item $pos$ est la position de la tête de lecture
	\end{itemize}
\end{definition}

Comment se déroule une exécution d'une machine de Turing pour un mot $w \in \Sigma^*$ ?

\begin{enumerate}
	\item On initialise le ruban
	      \begin{figure}[!htb]
		      \centering
		      % Inspired from https://tex.stackexchange.com/questions/49839/turing-machine-figure
		      \begin{tikzpicture}[every node/.style={block},
				      block/.style={minimum height=1.5em,outer sep=0pt,draw,rectangle,node distance=0pt}]
			      \node (A) {$w_0$};
			      \node (B) [left=of A] {$B$};
			      \node (C) [left=of B] {$\ldots$};
			      \node (D) [right=of A] {$\ldots$};
			      \node (E) [right=of D] {$w_n$};
			      \node (F) [right=of E] {$B$};
			      \node (G) [right=of F] {$\ldots$};
			      \node (T) [below = 0.75cm of A] {$q_i$};
			      \draw[-latex] (T) -- (A);
			      \draw (C.north west) -- ++(-1cm,0) (C.south west) -- ++ (-1cm,0)
			      (G.north east) -- ++(1cm,0) (G.south east) -- ++ (1cm,0);
		      \end{tikzpicture}
	      \end{figure}

	\item Si on est dans l'état $q$, que la lettre sous la tête de lecture est $a$ et que $\delta(q, a) = (q', b, d)$, alors on fait :
	      \begin{itemize}
		      \item On écrit $b$ à la place de $a$
		      \item On se déplace dans l'état $q'$
		      \item On déplace la tête de lecture dans la direction $d$
	      \end{itemize}
	\item Si on arrive dans l'état $q_a$ ou $q_r$, alors on arrête l'exécution. Si on arrive dans l'état $q_a$, alors on accepte le mot $w$ et si on arrive dans l'état $q_r$, alors on rejette le mot $w$.
\end{enumerate}


\begin{notation}
	Soit $M$ une machine de Turing, $w \in \Sigma^*$ un mot, alors on note $M(w)$ l'exécution de $M$ sur $w$. Cette exécution peut être :
	\begin{itemize}
		\item Acceptée : $M(w) = 1$
		\item Rejetée : $M(w) = 0$
		\item Bouclée : $M(w) = \bot$
	\end{itemize}
\end{notation}


\begin{remarque}
	Les automates s'injectent dans les machines de Turingm il suffit d'ignorer le ruban et de considérer que la tête de lecture est fixe.
\end{remarque}

\begin{definition}[Machine de Turing non déterministe]
	De manière analogue aux automates la notion de \textbf{machine de Turing non déterministe} étend la définition d'une machine de Turing en
	permettent d'avoir plusieurs transitions pour un état donné. La différence se trouve donc dans le type de la fonction de transition :

	$$ \Delta: Q \times \Gamma \to \parts{Q \times \Gamma \times \{R, L, N\}}$$

\end{definition}


\subsection{Notion de calculabilité}


\begin{definition}[Langage semi-décidable]
	Un langage $L \subseteq \mots$ est \textbf{semi-décidable} s'il existe une machine de Turing $M$ \tlq
	$$ \forall w \in \mots, w \in L \iff M(w) = 1 $$
\end{definition}

\begin{definition}[Langage décidable]
	Un langage $L \subseteq \mots$ est \textbf{décidable} s'il existe une machine de Turing $M$ \tlq
	$$ \forall w \in \mots, w \in L \implies M(w) = 1 \quad \text{et} \quad w \notin L \implies M(w) = 0 $$
	et $M$ s'arrête pour tout $w$.
\end{definition}

\begin{prop}
	Tout langage décidable est semi-décidable.
\end{prop}

\begin{proof}
	Il suffit de monter que si $M$ est une machine de Turing qui décide $L$, alors $M(w) = 1 \iff w \in L$.
	\begin{itemize}
		\item $w \in L \implies M(w) = 1$ est vrai par définition.
		\item $M(w) = 1 \implies w \in L$ peut être montrée par contraposée. Si $w \notin L$, alors $M(w) = 0$ car $M$ décide $L$ et donc $M(w) \neq 1$.
	\end{itemize}
\end{proof}

\begin{definition}[Fonction calculable]
	$f : \mots \to \mots$ est calculable si $\exists M$ \tq $\forall w \in \mots$ $M$ s'arrête sur $w$ avec $f(w)$ sur le ruban.
\end{definition}

\begin{lemme}
	La fonction $succ : \mots \to \mots$ est calculable.
\end{lemme}

\begin{prop}
	$L$ est décidable $\iff$ sa fonction caractéristique est calculable.
\end{prop}

\begin{definition}
	$f : \mots \to \mots$ énumère $L \subseteq \mots$ si $\im f = L$ \ie $\forall w \in L \iff \exists w' \in \mots, f (w') = w$.
\end{definition}

\begin{prop}
	$L$ est récursivement énumerable $\iff$ $L$ est décidable.
\end{prop}

On peut encoder toute machine de Turin par un nombre, qui est appété le nombre de Gödel et noté : $\encode M \in \mots$.

\begin{definition}[Énumération des fonctions calculables]
	$\forall n \in \N, \ \phi_n$ est la fonction calculée par la $n$-ième machine de Turing.
	$$\phi_{\encode M} (w) = M (w)$$
\end{definition}

\subsection{Quelques problèmes}

Quelques problèmes décidables en rapport aux automates :

\begin{itemize}
	\item $\text{ACCEPT}_A = \setdef {<A,w>} {A \text {est un automate qui accepte } w}$.
	      Dire que c'est problème est décidable revient a dire que $\exists$ un interpréteur d'automates en Machine de Turing.
	\item $\text{EQUIV}_A = \setdef {<A,A'>} {A \text { et } A' \text{ acceptent le meme langage}}$.
	\item $\text{EXISTS}_A = $ Il existe un mot reconnu.
	\item $\text{INFINITE}_A = $ l'automate est infini.
\end{itemize}

Cependant, ces problèmes étendus aux automates a piles, ne restent pas tous décidable.
$\text{ACCEPT}_{A_p}$ et $\text{INFINITE}_{A_p}$ restent décidables mais pas $\text{EQUIV}_{A_p}$.

\begin{definition}[Problème de l'arrêt]
	Le problème de l'arrêt est définit comme suit $HALT = \setdef {\encode{M,w}} {M \text{ s'arrête sur } w}$.
\end{definition}

\begin{lemme}[Machines universelles]
	Il existe une machine universelle $U$, \ie, $$\forall M,w, \  U (\encode {M,w}) = M (w)$$
	ou de manière equivalente
	$$\exists u, \forall n,w, \ \phi_n(w) = \phi_u(\encode {n,w})$$
	lemme\end{lemme}

\begin{proof}
	Admisse.
\end{proof}

\begin{prop}
	$HALT$ est semi-décidable.
\end{prop}

\begin{proof}
	Il suffit d'écrire un programme "impératif" :

	Entrée : $\encode {M,w}$

	Code:
	u(M,w);
	return 1;

\end{proof}

\begin{prop}
	HALT est indécidable.
\end{prop}

\begin{proof}
	Supposons par l'absurde que HALT est décidable. Alors il existe un entier $n$ \tq $\phi_n$ décide HALT, c'est-à-dire
	$ \phi_n(\encode {M, w} ) = 1 \iff \phi_n(w) \neq \bot $ et vaut $0$ sinon.

	Soit $n'$ le code de la fonction qui sur $w$ vaut 1 si $\phi_n (w,w) = 0 $ et n'est pas définie sinon. Alors
	\begin{eqnarray*}
		\phi_{n'}(n') = 1 & \text{si} & \phi_n(n',n') = 0\\
		&\text{\ie}& \phi_{n'}(n') \text{ ne s'arrête pas} \\
		&\text{\ie}& \phi_{n'}(n') = \bot
	\end{eqnarray*}


	\begin{eqnarray*}
		\phi_{n'}(n') = \bot & \text{si} & \phi_n(n',n') = 1\\
		&\text{\ie}& \phi_{n'}(n') \neq \bot \quad \lightning
	\end{eqnarray*}

\end{proof}

\section{Notion de calculabilité}


\begin{definition}[Langage semi-décidable]
	Un langage $L \subseteq \mots$ est \textbf{semi-décidable} s'il existe une machine de Turing $M$ \tlq
	$$ \forall w \in \mots, w \in L \iff M(w) = 1 $$
\end{definition}

\begin{definition}[Langage décidable]
	Un langage $L \subseteq \mots$ est \textbf{décidable} s'il existe une machine de Turing $M$ \tlq
	$$ \forall w \in \mots, w \in L \implies M(w) = 1 \quad \text{et} \quad w \notin L \implies M(w) = 0 $$
	et $M$ s'arrête pour tout $w$.
\end{definition}

\begin{prop}
	Tout langage décidable est semi-décidable.
\end{prop}

\begin{proof}
	Il suffit de monter que si $M$ est une machine de Turing qui décide $L$, alors $M(w) = 1 \iff w \in L$.
	\begin{itemize}
		\item $w \in L \implies M(w) = 1$ est vrai par définition.
		\item $M(w) = 1 \implies w \in L$ peut être montrée par contraposée. Si $w \notin L$, alors $M(w) = 0$ car $M$ décide $L$ et donc $M(w) \neq 1$.
	\end{itemize}
\end{proof}

\begin{definition}[Fonction calculable]
	$f : \mots \to \mots$ est calculable si $\exists M$ \tq $\forall w \in \mots, M$ s'arrête sur $w$ avec $f(w)$ sur le ruban.
\end{definition}

\begin{lemma}
	La composition de deux fonctions calculables est calculable
\end{lemma}

\begin{lemma}
	La fonction $succ : \mots \to \mots$ est calculable.
\end{lemma}

\begin{prop}
	$L$ est décidable $\iff$ sa fonction caractéristique est calculable.
\end{prop}

\begin{definition}
	$f : \mots \to \mots$ énumère $L \subseteq \mots$ si $\im f = L$ \ie $\forall w \in L \iff \exists w' \in \mots, f (w') = w$.
\end{definition}

\begin{prop}
	$L$ est récursivement énumerable $\iff$ $L$ est semi-décidable.
\end{prop}

\begin{proof}
    Vue en TD
\end{proof}


On peut encoder toute machine de Turing par un nombre, qui est appété le nombre de Gödel et noté : $\encode M \in \mots$.

\begin{definition}[Énumération des fonctions calculables] \label{def:enum}
	$\forall n \in \N, \ \phi_n$ est la fonction calculée par la $n$-ième machine de Turing.
	$$\phi_{\encode M} (w) = M (w)$$
\end{definition}


\begin{lemma}[Machines universelles]\label{lem:univ}
	Il existe une machine universelle $U$, \ie, $$\forall M,w, \  U (\encode {M,w}) = M (w)$$
	ou de manière equivalente
	$$\exists u, \forall n,w, \ \phi_n(w) = \phi_u(\encode {n,w})$$
\end{lemma}

\begin{definition}[eval]
	On note $eval(\encode M, w, n)$ la machine que simule l'exécution de $M$ sur $w$ en au plus $n$ étapes.
\end{definition}

\begin{prop}[Admis]
	$eval$ est décidable.
\end{prop}

\begin{prop}
	$L$ est décidable $\iff$ $L$ est semi-décidable \emph{et} $L^c$ est semi-décidable
\end{prop}

\begin{proof}
	\begin{itemize}
		\item \bimpLR \ triviale
		\item \bimpRL \\
		      On peut exécuter en parallèle des deux machines. Elle s'arrête forcement car on a bien que $\forall w, w \in L \lor w \notin L$.
	\end{itemize}
	Ce théorème et sa preuve détaillée correspondent à ceux présentés dans \cite[Theorem~4.22]{sipser}.
\end{proof}


\begin{prop}
	$L$ décidable $\iff \exists L_d \text{ décidable}, L = \setdef {w \in \mots} {\exists w' \in \mots, \encode{w,w'} \in L_d}$
\end{prop}


\begin{proof}
	$w \in L \iff \exists t, eval(\encode {M_L}, w, k) = 1$

	Alors on pose $L_d = \setdef {\encode{w, k}} {eval(\encode {M_L}, w,k) = 1}$.
	$L_d$ est décidable et on a l'équivalence par construction.
\end{proof}


\subsection{Quelques problèmes}

Quelques problèmes décidables en rapport aux automates :

\begin{itemize}
	\item $\text{ACCEPT}_A = \setdef {<A,w>} {A \text { est un automate qui accepte } w}$.
	      Dire que c'est problème est décidable revient a dire que $\exists$ un interpréteur d'automates en Machine de Turing.
	\item $\text{EQUIV}_A = \setdef {<A,A'>} {A \text { et } A' \text{ acceptent le meme langage}}$.
	\item $\text{EXISTS}_A = $ Il existe un mot reconnu.
	\item $\text{INFINITE}_A = $ le langage reconnu par $A$ est infini.
\end{itemize}

Cependant, ces problèmes étendus aux automates a piles, ne restent pas tous décidable.
$\text{ACCEPT}_{A_p}$ et $\text{INFINITE}_{A_p}$ restent décidables mais pas $\text{EQUIV}_{A_p}$.

\begin{definition}[Problème de l'arrêt]
	Le problème de l'arrêt est définit comme suit $HALT = \setdef {\encode{M,w}} {M \text{ s'arrête sur } w}$.
\end{definition}

\begin{proof}
	Admisse.
\end{proof}

\begin{prop}
	$HALT$ est semi-décidable.
\end{prop}

\begin{proof}
	Il suffit d'écrire un programme "impératif" :

	Entrée : $\encode {M,w}$

	Code:
	u(M,w);
	return 1;

\end{proof}

\begin{prop}
	HALT est indécidable.
\end{prop}

\begin{proof}
	Supposons par l'absurde que HALT est décidable. Alors il existe un entier $n$ \tq $\phi_n$ décide HALT, c'est-à-dire
	$ \phi_n(\encode {M, w} ) = 1 \iff \phi_n(w) \neq \bot $ et vaut $0$ sinon.

	Soit $n'$ le code de la fonction qui sur $w$ vaut 1 si $\phi_n (w,w) = 0 $ et n'est pas définie sinon. Alors
	\begin{eqnarray*}
		\phi_{n'}(n') = 1 & \text{si} & \phi_n(n',n') = 0\\
		&\text{\ie}& \phi_{n'}(n') \text{ ne s'arrête pas} \\
		&\text{\ie}& \phi_{n'}(n') = \bot\\\\
		\phi_{n'}(n') = \bot & \text{si} & \phi_n(n',n') = 1\\
		&\text{\ie}& \phi_{n'}(n') \neq \bot \quad \contradict
	\end{eqnarray*}
\end{proof}

\begin{exemple}
	Le Problème de Correspondance de Post (PCP) est indécidable. Le détail peut être consulté dans \cite[Chapter~5.2]{sipser}
\end{exemple}

\section{Réductions}

\subsection{Réduction many-one}

\begin{definition}[Réduction many-one]
	$A \leqm B$ \ssi $\exists \fmots f$ calculable et totale \tlq
	$$\forall w \in \alphabet, w \in A \iff f(w) \in B$$
\end{definition}


\begin{theorem} \label{thm:leqm_dec}
	$A \leqm B$ et $B$ est décidable, alors $A$ est décidable
\end{theorem}

\begin{proof}

	Soit $M_B$ la machine qui reconnait $B$ et $M_f$ celle qui calcule $f$. Alors
	\begin{eqnarray*}
		\forall w \in \mots, w \in A &\iff& f(w) \in B \\
		&\iff& M_B (M_f (w)) = 1 \\
		&\iff& M_A \text{ accepte } w
	\end{eqnarray*}

	et de manière analogue

	\begin{eqnarray*}
		\forall w \notin \mots, w \notin A &\iff& f(w) \notin B \\
		&\iff& M_B (M_f (w)) = 0 \\
		&\iff& M_A \text{ rejette } w
	\end{eqnarray*}


\end{proof}


\begin{coro}
	$A \leqm B$ et $A$ n'est pas décidable, alors $B$ n'est l'est pas non plus.
\end{coro}

\begin{proof}
	C'est la contraposée du théorème précédent.
\end{proof}

\begin{theorem}
	$A \leqm B$ et $B$ est semi-décidable, alors $A$ est semi-décidable
\end{theorem}

\begin{proof}
	Voir \ref{thm:leqm_dec}
\end{proof}

\begin{theorem}
	$A \leqm B$ et $B$ est co-semi-décidable, alors $A$ est co-semi-décidable.
\end{theorem}

\begin{proof}
	Si $A \leqm B$ alors $\overline A \leqm \overline B$ car
	\begin{eqnarray*}
		w \notin A &\iff& f(w) \notin A \\
		&\iff& A \leqm B
	\end{eqnarray*}
\end{proof}

\begin{definition}
	$A \equivm B \iff A \leqm B \et B \leqm A$
\end{definition}

\subsection{Classes d'équivalence sous many-one}

\begin{definition}[Many-one complet]
	$L$ est many-one complet pour une classe de langages $\mathcal C$ si $\forall L' \in \mathcal C, L' \leqm L$
\end{definition}

\begin{prop}[Problème de Post]
	HALT est RE-complet.
\end{prop}

\begin{proof}
	Soit $L \in $ RE, montrons que $L \leqm \halt$. Soit $M_L$ la machine qui semi-décide $L$.

	Alors on construit $M'_L$ la machine suivante:
	\begin{algorithmic}[lines]
		\Function{$M'_L$}{$\encode w$}
		\If {$M_L (w)$} {$1$}
		\Else { $\bot$}
		\EndIf
		\EndFunction
	\end{algorithmic}

	\begin{eqnarray*}
		w \in L &\iff& M_L = 1 \\
		&\iff& M'_L \neq \bot \\
		&\iff& M'_L \text{ s'arrête sur } w\\
		&\iff& \encode{M'_L,w} \in \halt
	\end{eqnarray*}

	Et donc, on pose $f (w) = \encode {M'_L,w}$ et on a bien que
	$w \in L \iff f(w) \in \halt$ et donc HALT est RE-complet.
\end{proof}

\begin{exercice}
	Construire, à partir de HALT, un langage qui n'est ni RE ni co-RE.
\end{exercice}

\begin{proof}
	Considérons
	$$ L = \setdef {\encode {M,M',w}} {M(w) = \bot \et M'(w) \neq \bot} $$

	Comme HALT et $\overline {\halt}$ sont RE-complet et co-RE-complet respectivement, il suffit de montrer $\halt \leqm L \et \overline{\halt} \leqm L$.

	\begin{itemize}
		\item $\encode {M,w} \in \halt \iff \encode {M_{loop}, M, w} \in L$. Donc $\halt \leqm L$.
		\item $\encode {M,w} \in \overline{\halt} \iff \encode {M, M_1, w} \in L$. Donc $\overline{\halt} \leqm L$ (où $M_1$ est une machine qui s'arrête toujours).
	\end{itemize}

	On a donc que $L$ n'est ni RE ni co-RE.
\end{proof}



\section{Théorèmes de récursion}


\begin{theorem}[d'itération / Smn / d'application partielle]\label{thm:it}
	Il existe une fonction calculable et totale $s$ \tlq
	$$\forall n,m,w, \phi_{s(m,n)}(w) = \phi_m(\encode {n,w})$$
	Si $m$ est le code d'un programme et $n$ un mot, alors $s(m,n)$ est le code de l'application partielle de $\phi_m$ à $n$.
\end{theorem}

\begin{theorem}[de point fixe]
	Si $ \fmots f$ \emph{calculable} et \emph {totale}. Alors il existe $e \in \mots$ \tq $\phi_e = \phi_{f(e)}$.
\end{theorem}


\begin{proof}
	Soit $G$ la machine : $(x,y) \to \letin e {\universal x x} {\universal e y}$.

	On pose $h(x) = s(\encode G, x)$ (le $s$ du théorème precedent). On a que $f \circ h$ est calculable et totale et notons son code $c$. Alors

	\begin{eqnarray*}
		\phi_{h(c)} (w) &=& \phi_{s(\encode G, c)} (w) \reason{par définition de $h$} \\
		&=& \phi_{\encode G} (\encode {c, w}) \reason{par \ref{thm:it}} \\
		&=& G (\encode {c, w}) \reason{correspondence énumeration machine }\\
		&=& \letin e {\universal c c} \universal e w \reason{par définition de $G$ }\\
		&=& \phi_{f \circ h (c)}(w) \reason{car $\universal c c =_{\ref{lem:univ}} \phi_c(c) =_{\ref{def:enum}} f \circ h (c)$}
	\end{eqnarray*}
	Donc $\phi_{h(c)} = \phi_{f \circ h (c)}$ et donc $h(c)$ est notre point fixe.
\end{proof}


\begin{theorem}[de récursion]
	Si $f : \mots \times \mots \to \mots$ est une fonction partielle et calculable. Alors il existe une machine $R$ qui calcule $\fmots r $ \tq
	$$ \forall w, r(w) =  f (\encode R, w)$$

	Autrement dit, $\exists e, \phi_e (w) = f (e, w)$
\end{theorem}


\begin{proof}
	Soit $M_f$ la machine qui calcule $f$ et $\fmots g, \ g(p) = s(\encode {M_f}, p)$
	Alors on applique le théorème de point fixe à $g$ et on a qu'il existe $r$ \tq $\phi_r(w)
		=\phi_{s(\encode{M_f},r)} (w) = \phi_{\encode {M_f}} (r,w) = f(r,w)$
\end{proof}


\begin{theorem}[de Rice]
	Toute propriété non triviale relative au langage reconnu par une machine de Turing est indécidable.

	Autrement dit, soit $L = \setdef {\encode M} {P(L_M)}$, avec $P$ une propriété non triviale, \ie $\exists M_1$ \tq
	$\encode {M_1} \in L$ et $\exists M_2$ \tq $\encode{M_2} \notin L$. Alors $L$ n'est pas décidable.
\end{theorem}


\begin{proof}
	Soit $L = \setdef {\encode M} {P(L_M)}$.
	Sans perte de généralité, supposons $P(\emptyset)$, \ie $\encode {M_{loop}} \in L$.
	On peut faire ceci car, dans le cas ou $\lnot P (\emptyset)$, alors on considère $\bar L$.

	Alors $\encode {M_{loop}} \in L \et \exists \encode{M_2} \notin L$, car $P$ est non triviale.

	Montrons que $\overline {\halt} \leqm L$.
	Soit $M$ une machine de Turing, pour tout $w$ on pose
	$$M' = \fun u {M(w); M_2(u)}$$
	Alors
	\begin{eqnarray*}
		M(w) = \bot &\iff& \lang M = \emptyset \\
		&\iff& M_L (M') = 1
	\end{eqnarray*}
	Donc $\overline {\halt} \leqm L$ et ainsi $L$ n'est pas décidable.
\end{proof}



\begin{prop}
	Il existe un Quine, \ie une machine $M$ \tq :
	$$\forall w, M(w) = \encode M$$
\end{prop}

\begin{proof}
	On applique le théorème de récursion avec la fonction $f(e, \_) = e$. Donc $\exists R, \, R(w) = f(\encode R, w) = \encode R$.
\end{proof}


\section{Hiérarchie arithmétique}


%TODO: Add 
Formules arithmétiques du premier ordre sous forme prénexe.


\begin{definition}[Saut de Turing]
	$X$ est un langage,

	$X' = \setdef c {\phi_c^X(c) \text{ est défini}} = \setdef {\encode M} {M^X \text{ s'arrete sur } \encode M}$

	$\phi_{c \in \mots}$ : énumeration des fonctions calculables.

	$\phi_{c \in \mots}^X$ : énumeration des fonctions calculables relativement a $X$. Et donc $\phi_{\encode M}^X(w) = M^X(w)$
\end{definition}


\begin{exemple}
	$$\emptyset' = \setdef {\encode M} {M^{\emptyset} \text{ s'arrete sur } \encode M} \equivm \halt$$
\end{exemple}

\begin{exercice}
	Montrer que $\emptyset '$ est $\Sigma_1$-complet :
	\begin{enumerate}
		\item $\emptyset' \in \Sigma_1$
		\item $\forall L \in \Sigma_1, L \leqm \emptyset'$
	\end{enumerate}
\end{exercice}


\begin{lemma}
	Si $A$ est $C$-complet:  $\exists Y$ tel que $X$ est $re(Y)$ avec $Y \in C$ $\iff$ $X$ est $re(A)$.
\end{lemma}

\begin{theorem} [de post]
	Nous avons les résultats suivants :
	\begin{enumerate}
		\item
		      \begin{enumerate}
			      \item
			            \begin{eqnarray*}
				            L \in \Sigma_{n+1} &\iff& L \text{ est r.e. relativement  un langage }  \Pi_n  \\
				            &\iff& L \text{ est r.e. relativement  un langage }  \Sigma_{n}
			            \end{eqnarray*}

			      \item
			            \begin{eqnarray*}
				            L \in \Pi_{n+1} &\iff& L \text{ est co-r.e. relativement  un langage }  \Sigma_n  \\
				            &\iff& L \text{ est r.e. relativement  un langage }  \Pi_n
			            \end{eqnarray*}
		      \end{enumerate}

		\item Il existe un langage $\Sigma_n$ complet, noté $\emptyset^{(n)}$.
	\end{enumerate}
\end{theorem}

TODO






\newpage

\bibliographystyle{plainnat}
\bibliography{calculabilite}


\end{document}

